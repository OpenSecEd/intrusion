\documentclass[a4paper,nocourse]{miunasgn}
\usepackage[utf8]{inputenc}
\usepackage[T1]{fontenc}
\usepackage[swedish,english]{babel}
\usepackage[hyphens]{url}
\usepackage{hyperref}
\usepackage[today,nofancy]{svninfo}
\usepackage{listings} 
\usepackage{varioref,prettyref}
\usepackage[natbib,style=numeric-comp,maxbibnames=99,sorting=none]{biblatex}
\addbibresource{literature.bib}
\usepackage[listings,varioref,prettyref]{miunmisc}

\svnInfo $Id$

\courseid{DT145G}
\course{Computer Security}
\assignmenttype{Laboratory Assignment}
\title{Host-Based Intrusion Detection}
\author{Daniel Bosk\footnote{%
  This work is licensed under the Creative Commons Attribution-ShareAlike 3.0 
  Unported license.
	To view a copy of this license, visit 
	\url{http://creativecommons.org/licenses/by-sa/3.0/}.
}}
\date{\svnId}

\begin{document}
\maketitle
\thispagestyle{foot}
\tableofcontents

\section{Introduction}
\label{sec:intro}
This laboratory exercise will cover the topic of intrusion detection
systems (IDS).
An IDS monitors the activity of a system and alert the administrator if any 
potential threats are detected.
This laboratory work will focus on host-based intrusion detection systems 
(HIDS), in particular OSSEC\footnote{%
  URL: \url{http://www.ossec.net/}.
}.


\section{Aim}
\label{sec:aim}
After completion of this assignment you will
\begin{itemize}
    \item Analyse the functionality of a host-based intrusion detection system.

\end{itemize}

This laboratory assignment will cover the open source HIDS OSSEC.


\section{Reading instructions}
\label{sec:reading}
Before starting this assignment you should have first read chapter 10 ``Banking 
and Book-keeping'' in \emph{Security Engineering} \cite{Anderson2008sea} 
followed by section 17.4 in \emph{Computer Security} \cite{Gollmann2011cs}.
Then you should read chapters 2-4 in \emph{OSSEC HIDS: Host-based Intrusion 
Detection Guide} \cite{ossec2,ossec3,ossec4}, covering how OSSEC works.



\section{Work}
\label{sec:tasks}
This section covers the work to be done and the next section covers how it will 
be examined, and what to be done to pass it.

Firstly you should install the OSSEC software.
You can find it on URL
\begin{center}
  \url{http://www.ossec.net/}.
\end{center}
The installation procedure is documented both on the website and in 
\cite{ossec2}, the instructions on the Web are of course more up-to-date.

Once the system is installed and up-and-running, you should select one of the 
features it was designed for.
You will now evaluate this feature and prepare a demonstration of it.
This demonstration should present at least the following:
\begin{itemize}
  \item What is interesting about this feature?
  \item How you evaluated the feature.
  \item A demonstration of its use.
\end{itemize}


\section{Examination}
\label{sec:examination}
As you will prepare a demonstration, this will be presented for the class.
You are required to have some slides to present the feature and your 
evaluation, and then of course the live demonstration.

In summary you should do this:
\begin{itemize}
  \item You should hand in the slides (in PDF-format) in the course platform.
  \item You should also give your presentation (demonstration), it should be at 
    most 15 minutes long.
\end{itemize}


\printbibliography
\end{document}
