\documentclass[a4paper,nocourse]{miunasgn}
\usepackage[utf8]{inputenc}
\usepackage[T1]{fontenc}
\usepackage[swedish,english]{babel}
\usepackage{url,hyperref}
\usepackage{natbib}
\usepackage[today,nofancy]{svninfo}
\usepackage{listings} 
\usepackage{varioref,prettyref}
\usepackage[natbib,listings,varioref,prettyref]{miunmisc}

\svnInfo $Id$

\courseid{DT116G}
\course{Network Security}
\assignmenttype{Laboratory Assignment}
\title{Intrusion Detection}
\author{Daniel Bosk\footnote{%
        E-post: \protect\url{daniel.bosk@miun.se}.
        } and
        Lennart Franked\footnote{%
            E-post: \protect\url{lennart.franked@miun.se}.
        }
}
\date{\svnId}

\begin{document}
\maketitle
\thispagestyle{foot}
\tableofcontents

\section{Introduction}
\label{sec:Introduction}
This laboratory exercise will cover the topic of intrusion detection
systems (IDS). An IDS monitors the activity of a system and alert the 
administrator if any potential threats are detected. In this course we are 
going to focus solely on Network Intrustion Detection Systems (NIDS).

Before starting this assignment you should have completed the laboratory
assignment covering GNU Privacy Guard (GPG).

\section{Aim}
\label{sec:Aim}
After completion of this assignment you will
\begin{itemize}
    \item Analyse the functionality of a host-based intrusion detection system.

\end{itemize}

This laboratory assignment will cover the open source NIDS Snort.

\section{Reading instructions}
\label{sec:Readinginstructions}
Before starting this assignment you should have first read chapter 10 ``Banking 
and Book-keeping'' in \emph{Security Engineering} \cite{Anderson2008sea} 
followed by section 17.4 in \emph{Computer Security} \cite{Gollmann2011cs}.
Then you should read chapters 2-4 in \emph{OSSEC HIDS: Host-based Intrusion 
Detection Guide} \cite{ossec2,ossec3,ossec4}, covering how OSSEC works.


\section{Tasks}
\label{sec:Tasks}
This section contains the task that you must perform in order pass this
laboratory assignment.

\subsection{Installation}
\label{subsec:Install}
Visit the Snort webpage and download the Snort installation files together with
the Snort rules package.

\begin{center}
\url{http://www.Snort.org}
\end{center}

Once installed add the Snort rules files to your working installation. 
Select one rule and use appropriate methods to test and see if the rule works.

When you have confirmed that the rules are working, You shall solve \emph{one} 
of the scenarios given below. Consult the documentation \citep{snort} for help 
on how to solve it. Note that not all scenarios can be solved solely with the
help of Snort.

\subsection{Scenario 1: Online Exam}

In a computer networking course that is given at Mid Sweden University, the
students take exams by logging in to a specific web site and access the exam
there. You have been given the task to, with the help of Snort, create a series
of rules that will alert the teacher if a student is accessing another web page
during the exam. Since the students do not want to be disturbed during the exam,
it is of vital importance that the rules will not result in any false
positives. You can use Cisco Netspace if you have an account there, otherwise,
use any online test available on the internet.

\subsection{Scenario 2: Snort as an IDPS}

You are tired of all the reconnaissance attacks that your server have been
the victim of lately, therefore you decide to with the help of Snort, block all
traffic from a destination that have performed such an attack (you can self 
decide what specific type of reconnaissance attack you have been the victim of).

\subsection{Scenario 3: Snort Notification}

You are running a server that is connected to the public network and would like
to get notified when this server is a victim of an attack without having to go
through the system logs. You will therefore configure Snort to notify you
either using email or by SNMP whenever a potential attack is detected.
(You can self decide what these attacks can be). As an optional task, you also
would like to create statistical data on how often your server is being
attacked. The data must be presented in such a way that you can, per attack, 
see how many times per hour and per day this attack occurs. It should be 
represented either in a text file that can easily be parsed to a spreadsheet 
program, or in a format readable to for example MRTG. 

\section{Examination}
\label{sec:examination}
The following results must be handed in.
\begin{itemize}
    \item Give a short summary of your installation process, including adding
      and testing of the rules.

    \item Give a detailed description on how you solved the scenario, Include
      your rule(s), together with a detailed description of how it (they)
      works, also a description of how you tested it and the resulting entry in the
      log file.
\end{itemize}

\subsection{Bonus}
  You can receive a 2 point bonus on the first exam in this course by
  solving Scenario 3. To get the bonus, the passed version of the 
  laboratory assignment must be submitted in time.

\bibliography{literature}

\end{document}
